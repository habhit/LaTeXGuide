\subsection{参考文献リスト}

レポートや論文などを書くときには,まったく何も参考にせずに,ゼロから書き上げることはまず起こりえない.必ず何らかのを文献を参考にし,そこから得た知識を利用して,自らの考察・提案をまとめていくことになる.どの部分を参考にし,どの部分が独創的な部分なのかを明確にするためにも,参考文献リストを明示し,それが本文のどの部分に関与しているのかを明らかにしておく必要がある.

参考文献リストを作成するために,\LaTeX には様々な便利な機能が備わっているが,ここではもっとも単純なものを紹介しておく\footnote{\BibTeX というツールを使えば,あらかじめ用意しておた文献データベースから自動的に整形して参考文献リストを作成できる.\BibTeX については基礎ゼミ2で扱う.} .

参考文献リストは文章の末尾に以下のように記述しておく.
\begin{screen}
\begin{verbatim}
\begin{thebibliography}{9}
   \bibitem{TeXText} 渡辺徹,好き好き \LaTeXe 初級編,2006.
   \bibitem{Kagishippo} 「物理のかぎしっぽ」\TeX に関するメモ,2018年5月21日閲覧.
\end{thebibliography}
\end{verbatim}
\end{screen}
この例は本書の末尾にある参考文献リストとほぼ同じもの出力するためのソースである\footnote{Webアドレスを書くと長くなるので省略している.}.簡単にその内容を説明する.
\begin{itemize}
\item \verb|\begin{thebibliography}| のあとにある \verb|{9}| の数値は参考文献の数を示している.もし参考文献の数が一桁であれば 9 ,二桁であれば 99 のように記述する.
\item \verb|\bibitem| で始まる項目がそれぞれの参考文献を指している.\verb|\bibitem| は直後に参照キーを指定する.本文ではこの参照キーを使って参考文献を指し示す.
\end{itemize}
このような参考文献リストを用意したあとで,本文中では \verb|\cite| コマンドを使って参考文献について言及する.たとえば,
\begin{screen}
\begin{verbatim}
情報実習Iでは教科書\cite{TeXText}を指定しているが,教科書は非常に細かに記述されているので,いきなり教科書を読んでも理解は難しいかもしれない.
\end{verbatim}
\end{screen}
とソースファイルに書けば,
\begin{screen}
情報実習Iでは教科書\cite{TeXText}を指定しているが,教科書は非常に細かに記述されているので,いきなり教科書を読んでも理解は難しいかもしれない.
\end{screen}
と出力される.参照キーを用いれば,それが何番の参考文献に相当するかは \LaTeX が自動的に判断して番号に置換される.
