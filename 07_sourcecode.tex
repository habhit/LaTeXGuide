\subsection{ソースコードの挿入}

プログラムを作成するレポートでは,作成したソースコードを示すためにレポートに挿入することがある.
\ref{sec:return}節で述べたように,\LaTeX では単独の改行は無視されるので,
ソースコードをそのまま挿入しても正しく出力されない.
ソースコードを見やすく挿入する方法はいくつかあるが,
ここでは \verb+ moreverb + パッケージを使った方法を紹介する.

まず,\verb+ moreverb + パッケージを使うために,
プリアンブルに以下のように \verb+ \usepackage + 命令を追記する.
\begin{screen}
\begin{verbatim}
\usepackage{moreverb}
\end{verbatim}
\end{screen}
その上で,以下のように \verb+ listing + 環境の中にソースコードを書く.
\begin{screen}
   \begin{verbatim}
\begin{listing}{1}
/**
 * Hello, World と表示するサンプル
 */
public class HelloWorld {
   public static void main(String[] args) {
      Systen.out.println("Hello, World");
   }
}
\end{listing}
\end{verbatim}
\end{screen}
そうすると,以下のように各行の冒頭に行番号をつけてソースコードが出力される.
行番号が付くとレポート文中でソースコードの内容について言及する際に便利である.
\begin{screen}
\begin{listing}{1}
/**
 * Hello, World と表示するサンプル
 */
public class HelloWorld {
   public static void main(String[] args) {
      Systen.out.println("Hello, World");
   }
}
\end{listing}
\end{screen}

