% はじめに
\section{はじめに}

\LaTeX は理工系の論文・書籍などの出版に広く利用されている
文書処理(組版処理)システムである.
一般に使われている Word などのいわゆるワープロソフトとは使い勝手が異なるので,
最初は戸惑うこともあるが,慣れてくると簡単に読みやすい文章を作成することができる.

情報実習Iでは教科書\cite{TeXText}を指定しているが,教科書は非常に細かに記述されているので,
いきなり教科書を読んでも理解は難しいかもしれない.
本書は教科書を読む前準備として概要を解説するものである.
本書で概要を把握した上で教科書を読み進めていけば効率的に学修できるものと期待している.

なお,この文書自体も \LaTeX で作成している.
ソースファイルは GitHub\footnote{\url{https://github.com/habhit/LaTeXGuide}}
で公開するので,興味があれば読んでみて欲しい.
GitHub はソースコードなどを共有・管理するサービスである.
オープンソースのソフトウェアの公開などに広く利用されている.

\section{\TeX と \LaTeX}
\label{sec:TeXandLaTeX}

\TeX は1978年に数学者の Donald E. Knuth によってリリースされた
組版システムである\footnote{書籍などの紙面を構成する文字や図版を配置する工程のことを
組版と呼ぶ.数学者である Knuth が組版システムを開発するに至った経緯など
興味深い逸話がある.詳細は Wikipedia にも書かれている.}.
無料で使えるフリーソフトとしては配布されている.TeXを利用して組版するためには,
文章とともに,文章の見栄え・体裁に関する指定を記述する.
これは Web ページ(いわゆるホームページ)に使われている HTML 言語と同様であり,
マークアップ言語と呼ばれている.

\TeX 単体では使いにくい面があったので,
より手軽に使用できるように機能拡張された文書処理システムが \LaTeX である.
現在では \TeX 単体が利用されることは少なく,
ほとんどの場合は \LaTeX か \LaTeX をもとに機能拡張された文書処理システムを
使用することがほとんどである.

\LaTeX の特徴は以下のようなものがあげられる.
\begin{itemize}
\item そのまま出版できるような綺麗な文書が作成できる.
\item 数式を簡単に入力できる\footnote{最近では,MS Word などでも数式を
簡単に入力できるようになってきているが,
そこでは \LaTeX にならった記法が使われるので,
\LaTeX に習熟していれば Word での数式入力にも役立つ.}.
\item 様々なOSで同じように利用できる.
\end{itemize}
後で解説するように,出力イメージを確認しながら編集するものではないので
最初は戸惑うことも多いが,逆に,文章の内容に集中できる点が利点とも言えるので,
慣れると使いやすくなる.

