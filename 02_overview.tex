\section{\LaTeX による文書作成の流れ}

文書作成の流れは以下のようになる.

\begin{description}
\item[(A)] ソースファイルの編集
\item[(B)] タイプセットとPDFへの変換
\item[(C)] プレビューや印刷
\end{description}

まず,ソースファイルを作成・編集する(A).ソースファイルは文書本体と,
見栄えを指定するコマンドが書かれたものである.
\LaTeX の場合のソースファイルは \verb|.tex| という拡張子を付けるのが標準的である.
ソースファイルの簡単な例を\ref{sec:source}節に示している.これは単純なテキストファイルであるので,好みのソフトウェアを使って編集していけば良い.

次に, タイプセット(組版)とPDFへの変換を行う(B).
ここが \LaTeX による処理の部分であり,
ソースファイルを解釈して組版(文書や図版の書面への配置)を行う.
組版の結果は PDF ファイルとして出力する\footnote{PDF ファイルは
文書を配布する形式として広く利用されている.\LaTeX による組版
結果は dvi ファイルとして出力されているが,dvi ファイルの内容を確認するには
特別なソフトが必要であるので,最近では PDF に変換することが多い.}.

最後に PDF ファイルを表示・印刷して内容を確認する(C).
最近のOSでは,PDF ファイルを表示・印刷するためのソフトウェアは
標準的にインストールされていることが多い.

ここに示した(B)や(C)の手順で問題が生じれば,
(A)に戻ってソースファイルを修正することになる.

\section{\LaTeX ソースファイル}
\label{sec:source}

前節で述べた \LaTeX ソースファイルの例を以下に示す.

\begin{screen}
\begin{verbatim}
\documentclass{jsarticle}

\begin{document}

 こんにちは \LaTeX !!

\end{document}
\end{verbatim}
\end{screen}

ソースファイルについて,まずは以下の点だけは覚えておいて欲しい.

\begin{itemize}
\item \verb|\documentclass | はこれから書く文書の種類・
スタイルなどを指定するコマンドである.当面はあまり意識しなくても良い.
\item \verb|\documentclass | から \verb|\begin{document}| の間のことを
\textbf{プリアンブル}(前書き)と呼ぶ.
プリアンブルには追加機能を読み込むコマンドなど,文書全体に影響を及ぼすコマンドを書く.
\item \verb|\begin{document}| から \verb|\end{document}| の間に文章の本文と,
その構成や見栄えを指定するコマンドを記述する.

\end{itemize}

ソースファイルをタイプセットするための手順や,
うまく行かなかった場合の処置などについては実習で用いた解説スライド\cite{JJ1-3}を参照のこと.
上で示したソースファイルをタイプセットすると,以下のような文だけの文書が得られる.

\begin{screen}
こんにちは \LaTeX !!
\end{screen}

